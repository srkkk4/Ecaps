\documentclass[UTF8]{ctexart}
\usepackage{geometry}
\usepackage{fancyhdr}
\usepackage{graphicx}
\usepackage{array}
\usepackage{multirow}
\usepackage{ulem}
\newcommand{\PreserveBackslash}[1]{\let\temp=\\#1\let\\=\temp}
\newcolumntype{C}[1]{>{\PreserveBackslash\centering}p{#1}}
\newcolumntype{R}[1]{>{\PreserveBackslash\raggedleft}p{#1}}
\newcolumntype{L}[1]{>{\PreserveBackslash\raggedright}p{#1}}
% \author{ARZhu}
\author{\zihao{-3} round 1 }
\title{\zihao{2} Dyd の Test }
\date{}
\geometry{left=3.18cm,right=3.18cm,top=2.54cm,bottom=2.54cm}
\begin{document}
 
\maketitle
\thispagestyle{empty}
\begin{center}
\zihao{-3}\textbf{(请选手务必仔细阅读本页内容)}
\end{center}
 
\textbf{一、题目概况}
\begin{center}
\begin{tabular}{*{5}{|C{8em}}|}
\hline
    中文题目名称 & 石头门 & 回路 & 炼金术 & 甜甜圈 \\ \hline
    英文题目与子目录名 & door & fleury & alchemy & donut \\ \hline
    可执行文件名 & door & fleury & alchemy & donut \\ \hline
    输入文件名 & door.in & fleury.in & alchemy.in & donut.in \\ \hline
    输出文件名 & door.out & fleury.out & alchemy.out & donut.out \\ \hline
    每个测试点时限 & 5秒 & 1秒 & 1秒 & 10秒 \\ \hline
    内存上限 & 2048M & 521M & 256M & 1024M \\ \hline
    测试点数目 & 10 & 10 & 10 & 10 \\ \hline
    每个测试点分值 & 10 & 10 & 10 & 10 \\ \hline
    附加样例文件 & 有 & 有 & 有 & 有 \\ \hline
    结果比较方式 & \multicolumn{4}{|c|}{全文比较(过滤行末空格及文末回车)} \\ \hline
    题目类型 & 传统 & 传统 & 传统 & 传统 \\ \hline
 
\end{tabular}
\end{center}
 
\textbf{二、提交源程序程序名}
\begin{center}
\begin{tabular}{*{5}{|C{8em}}|}
\hline
    对于C++语言 & door.cpp & fleury.cpp & alchemy.cpp & donut.cpp \\ \hline
\end{tabular}
\end{center}
 
\textbf{三、编译选项}
\begin{center}
\begin{tabular}{*{5}{|C{8em}}|}
\hline
   对于C++语言 & \multicolumn{4}{|c|}{\ \ \ \ \ \ \ \ \ \ \ \ \ \ \ \ \ \ \ \ \ \ \ \ \ \ \ \ \ \ \ \ \ \ -O2\ -lm \ -std=c++14 \ -static \ \ \ \ \ \ \ \ \ \ \ \ \ \ \ \ \ \ \ \ \ \ \ \ \ \ \ \ \ \ \ \ \ \ } \\ \hline
\end{tabular}
\end{center}
 
\textbf{注意事项(请仔细阅读)}
\begin{enumerate}
    \item{文件名(程序名和输入输出文件名)必须使用英文小写。}
    \item{C/C++中函数main()的返回类型必须是int,程序正常结束时的返回值必须是0。}
    \item{若无特殊说明,结果比较方式为忽略行末空格、文末回车后的全文比较}
    \item{程序可使用的栈空间大小与该题内存空间限制一致}
    \item{每道题目所提交的代码文件大小限制为 $100KB$ }
    \item{(理论上来说)评测时采用的机器配置为:Inter(R) Core(TM) i7-8700K CPU @3.70GHz, 内存 32GB ,上述时限以此配置为准。评测在NOI Linux下进行。}
    \item{特别提醒:请注意各题目的时空限制}
\end{enumerate}
 
\newpage
\setcounter{page}{1}
\pagestyle{plain}
\pagenumbering{arabic}
 
% T1
\newpage
\section{石头门(door)}
\begin{center}
\end{center}
\subsection{问题描述}
 “一切都是命运石之门的选择”	——《命运石之门》

你打开了石头门,在前 $8$ 集磨了半天,忍不住开始想 $\alpha$ 线的克莉丝汀娜(牧瀬紅莉栖)在干嘛,于是为了加快这痛苦的剧情,你决定去帮她快点把 タイムマシン (Time Machine)造出来

时间机器的研究建立在对世界线的研究上,而世界线可以看作一棵由事件组成的有根树(事件树),为了简单,我们把事件用字符表示,且保证字符集为 $\{*, \&, \#, \$\}$ (不包括逗号),例如,下面就是一棵事件树:

\begin{figure}[h]
    \centering
    \includegraphics[width=0.5\linewidth]{pic/240fcf1b.png}
    \caption{事件树}
    \label{fig:enter-label}
\end{figure}

我们把它们从 $1$ 开始编号,而在研究事件树的过程,会发生以下事件:

1. 世界线发展:在给定的点 $id$ 后面加一个事件 $x$ ,编号就是当前最大编号加 $1$ ,特别的,若 $id$ 已经删除,则该点也会立刻被删除(但最大点编号会加 $1$ )

2. 世界线收束:给定点 $u, v$ ,你需要将求出它们的 $lca$ ,不妨记作 $w$ ,并设路径 $(w, u)$  长于 $(w, v)$ (如果一样长,令先输入的为 $v$ ),路径 $(w, v)$ 会按以下规则合并到 $(w, u)$ 上:

   从 $w$ 开始用指针 $i, j$ 分别向 $u, v$ 走(保证 $i, j$ 到 $w$ 距离相同),把当前点 $j$ 的事件加到点 $i$ 上(完成后 $i$ 上有两个点的所有事件),再把点 $j$  不在路径 $(w, v)$ 上的儿子都给 $i$ 

   合并完成后,把路径 $(w, v)$ 删除

   例如,上图中,我们合并 $5, 11$ ,结果为:

\begin{figure}[h]
    \centering
    \includegraphics[width=0.5\linewidth]{pic/qyKYTPZWON5ICvn.png}
    \caption{收束}
    \label{fig:enter-label}
\end{figure}

\newpage

   特别的,若 $w = u/v$ ,或者 $u/v$ 已经被删除,则不必进行任何操作 

3. 世界线变动:即在事件树上跳跃,但问题在于,你和克莉丝汀娜都不知道自己跳到了哪条线,所以你们决定分工合作,克莉丝汀娜按事件顺序为你提供当前线的事件,而你要定位当前线,若有多解,依次升序输出;定位规则如下:

   从任意一个节点开始,向儿子走,对于到达节点上的事件,如能和克莉丝汀娜给你的对应事件匹配(如节点上有多个事件,任意一个匹配即可),就继续走,最后走到的节点就是一个答案

   例如,她告诉你事件为 \$\$* ,那么你就要输出 9 11 (路径为 2-7-9 , 7-8-11 , 多个答案中间以空格隔开)

你能帮助她加快剧情发展吗?

\vspace{5mm}
简化题意

初始给定一颗 $n$ 个点的树,每个点上有一个字符集,初始时每个字符集中中只有一个字符

有 $m$ 次操作,分三类:

\begin{enumerate}
    \item{在编号为 $id$ 的点后面再加一个点,编号为当前最大编号 $+ 1$ ,初始字符集中有一个字符}
    \item{合并一条路径,具体合并方法见上面}
    \item{给定一个字符串,查询当前树上所有能与其匹配的路径(路径上每个点可以对应自己字符集中的任意\textbf{一个}字符)}
\end{enumerate}

\subsection{输入}
第一行一个整数 $n$ ,表示最初事件树的节点个数,为了简单,好心的出题人保证以 $1$ 为根

接下来一行 $n$ 个字符描述每个节点,为了简单,好心的出题人保证每个节点最初只有一个事件,这个保证对世界线发展得到的新节点同样适用

接下来 $n - 1$ 行每行两个整数 $u, v$ 描述树上一条无向边

接下来一行一个整数 $m$ 表示事件个数

接下来 $m$ 行,每行第一个整数 $opt$ 表示事件类型( $opt \in [1, 3]$ ),接下来的输入视操作而定:

\begin{itemize}
    \item 如果 $opt = 1$ ,输入一个整数和一个字符 $id, x$ 
    \item 如果 $opt = 2$ ,输入两个整数 $u, v$ 
    \item 如果 $opt = 3$ ,输入一个字符串 $s$ 
\end{itemize}

\subsection{输出}

对于每次世界线变动,输出若干整数表示答案,如没有,输出一个空行
 
\subsection{输入输出样例1}

见下发文件的 $door1.in/out$ 对应 $Sub1$ 

\subsubsection{输入样例}

\begin{verbatim}
10
# $ * & * # $ $ * &
1 2
2 3
1 4
4 6
4 5
2 7
7 8
7 9
7 10
4
1 8 *
3 $$*
2 5 11
3 &$&
\end{verbatim}

\subsubsection{输出样例}
\begin{verbatim}
9 11 
10 
\end{verbatim}

\subsection{输入输出样例2, 3, 4}
见下发文件
 
 
\subsection{约定和数据范围}

\begin{center}
\begin{tabular}{*{3}{|C{9em}}|}
\hline
    测试点编号 & 特殊范围 & 特殊性质 \\ \hline
    1, 2 & $n, m \le 100$ & 无 \\ \hline
    3, 4 & 无 & 保证只出现字符 \$ \\ \hline
    5, 6 & 无 & 保证任何时候,树是一条链 \\ \hline
    7 $\sim$ 10 & 无 & 无 \\ \hline
\end{tabular}
\end{center}

对于所有数据, $n, m \le 10^4$ ,保证字符集为 $\{*, \&, \#, \$\}$ ,每次询问的字符串长度 $\mid s \mid \le 50$ 

\begin{figure}[h]
    \centering
    \includegraphics[width=0.5\linewidth]{pic/4DV1oklWRhQ8n2K.png}
    \caption{石头门}
    \label{fig:enter-label}
\end{figure}


% T2
\newpage
\section{回路(fleury)}
\begin{center}
\end{center}
\subsection{问题描述}
“柴可夫斯基 B 小调,交响曲第 6 号,作品编号 74 【悲怆】” ——《玛莉亚†狂热》

Dyd 最近在看《玛莉亚†狂热》(\sout{符合我对四川的刻板印象}),众所周知,新房昭之的动画都有自己的特点,本番也不例外,这天鞠也(しどう まりや)为了捉弄宫前佳奈子(宫前かなこ /みやまえ)她丢到了一号旧宿舍,这个宿舍可以抽象成一个无向图,为了走出去宫前佳奈子需要找到一条欧拉回路,但她显然没有学过 OI ,所以她不会线性的做法,而只会避桥法

\vspace{5mm}
    避桥法:选择边走时,尽量选择\textbf{当前图}下不是桥的一条边走,如果没有,再走桥,可以证明在存在欧拉回路的情况下,这样是可以走完所有的边的
    \\
    注意走过后删边,删完以后图中的桥可能发生变化
\vspace{5mm}

在该算法中,如果走这条边并删除时,这条边并不是桥,那么宫前佳奈子就会认为这条边是“成功”的,她希望选择一个起点和合适的走边方法,看到有最多的“成功”的边,但她当然不会,所以希望你帮帮她(\textbf{请注意不一定原图存在欧拉回路,甚至图不一定联通})

\vspace{5mm}
简化题意

给定一张图,要求选择一个点,使得从这个点开始用避桥法求欧拉回路时,非桥边数量最多,输出最多的数量
\subsection{输入}
第一行两个整数 $n, m$,表示无向图的点数和边数

接下来 $m$ 行,每行两个整数 $u, v$,表示 $u$ 和 $v$ 之间有一条连边
\subsection{输出}
一行一个整数,表示最多的“成功”的边。
\subsection{输入输出样例1}
\subsubsection{输入样例}
\begin{verbatim}
3 3
1 2
2 3
1 3
\end{verbatim}
\subsubsection{输出样例}
\begin{verbatim}
1  
\end{verbatim}
\subsection{输入输出样例2, 3}
见下发文件

\subsection{约定和数据范围}

\begin{center}
\begin{tabular}{*{3}{|C{9em}}|}
\hline
    测试点编号 & 特殊范围 & 特殊性质 \\ \hline
    sub1 (5 pts) & $n\leq 10, m\leq 20$ & 无 \\ \hline
    sub2 (25 pts) & $n\leq 2000, m\leq 5000$ & 无  \\ \hline
    sub3 (10 pts) & 无 & 保证该图仅含有点不交的简单环 \\ \hline
    sub4 (20 pts) & 无 & 保证所有边在同一连通块 \\ \hline
    sub5 (40 pts) & 无 & 无 \\ \hline
\end{tabular}
\end{center}

对于全部数据,满足 $n\leq 5\times 10 ^ 5, m\leq 2\times 10 ^ 6$,保证所有点的度数为偶数

\begin{figure}[h]
    \centering
    \includegraphics[width=0.5\linewidth]{pic/vuyv9jyy.png}
    \caption{玛丽亚狂热}
    \label{fig:enter-label}
\end{figure}

(\sout{上图中有一个是男主角})

% T3
\newpage
\section{炼金术(alchemy)}
\begin{center}
\end{center}
\subsection{问题描述}
“人没有牺牲的话就什么也得不到,为了得到某些东西,就必须付出同等的代价,这就是炼金术的‘等价交换’原则,那是我们的坚信,那就是世界的真实”	——《钢之炼金术士》

as we all know ,小豆丁(爱德华·艾尔利克)因为右手与左脚为机械义肢而拥有“钢”之称号,是一位天才炼金术师,小小年纪就见过了真理之门(歪理之门),而你因为种种原因拥有“铜”之称号,今天,钢之炼金术士和铜之炼金术士决定一起炼金

虽然名字相似,但你们的炼金方法略有不同,炼金时你们会写下一串符文,小豆丁擅长炼钢,具体的,小豆丁毕竟是少年天才,他炼金靠一段由小写字母组成的咒语,只要符文中出现了他想要的咒语,他就可以炼出钢来;而你则略逊,擅长炼铜,炼成所用的咒语由大写字母组成,而且必须要保证你想要的咒语出现在指定位置(每个指定的位置都要出现),你才可以炼出铜来

现在你们打算一起炼制铜钢合金,由于配合不当,你们写了一串不知道是啥的符文(即符文的每个位置的字符在所有大小写字母中完全随机生成),在歪理之门还在思考该让你们炼出个啥时,你们开始思考你们炼制成功的概率了

认为你们炼制成功,当且仅当小豆丁成功炼出钢且你成功炼出铜

\vspace{5mm}
简化题意

有一个所有位置都随机的串,定义其是“成功”的,当且仅当:

1. 其中出现了小写串 $s$ 

2. 其在所有规定的位置都出现了大写串 $t$ 

现在,要求“成功”的概率

\subsection{输入}
第一行一个整数 $n$ ,表示你们写下的符文的总长度

接下来一行一个字符串 $s$ ,表示小豆丁的咒文

接下来一行一个个字符串 $t$ ,表示你的咒文

接下来一行一个整数 $k$ ,表示你的指定位置个数

接下来一行 $k$ 个整数,表示你指定的位置的开头
\subsection{输出}

一行一个数,表示你们炼制成功的概率,答案对 $998244353$ 取模

\subsection{输入输出样例1}
\subsubsection{输入样例}
\begin{verbatim}
5
ac
WA
1
2
\end{verbatim}
\subsubsection{输出样例}
\begin{verbatim}
888567007
\end{verbatim}
 
\subsection{输入输出样例2, 3, 4}
见下发文件

\subsection{约定和数据范围}

\begin{center}
\begin{tabular}{*{3}{|C{9em}}|}
\hline
    测试点编号 & 特殊范围 & 特殊性质 \\ \hline
    1, 2 & $n \le 5$ & 无 \\ \hline
    3, 4 & $\mid s \mid = 1$ & 无  \\ \hline
    5, 6 & $k = 0$ & 无 \\ \hline
    7 $\sim$ 10 & 无 & 无 \\ \hline
\end{tabular}
\end{center}

对于所有数据保证 $\mid s \mid, \mid t \mid \le \min(100, n)$ , $n \le 10^5$ , $k \le 20$ ,指定的位置 $\le n$ 且各不相同,保证字符都为大/小写字母

\begin{figure}[h]
    \centering
    \includegraphics[scale=0.3]{pic/ZWXcvgTfauU87ox.jpg}
    \caption{钢炼}
    \label{fig:enter-label}
\end{figure}


% T4
\newpage
\section{甜甜圈(donut)}
\begin{center}
\end{center}
\subsection{问题描述}
“曾互相伤害的我们,互舔伤口;
不再完美的我们,开始互相追求;
如果你的生命明天到了尾声,那我活到明天便足够;
如果你今天为我活了下来,那我也会活在当下”	——《伤物语》

小忍(姬丝秀忒·雅赛劳拉莉昂·刃下心,即 Kissshot·Acerolaorion·Heartunder Blade )虽然曾经是令人生畏的吸血鬼,但现在喜欢吃甜甜圈(当然主食还是血),为了讨好她,垃圾君(阿良良木历)决定给她买甜甜圈,但小忍的吃法很暴力,垃圾君觉得很丢主角的脸(明明他的发型更丢脸),所以垃圾君要限制她

垃圾君买来了 $n$ 个甜甜圈,每个甜甜圈的美味程度可以用 $k$ 个值来描述,第 $i$ 个记做 $a[i][1 \sim k]$ ,小忍吃第 $i$ 个甜甜圈的丢脸系数为 $v[i]$ 

垃圾君有时会买来新的甜甜圈替代原有的,而小忍有时会偷吃掉一些甜甜圈(因为她可以用能力把甜甜圈变回来,所以两次吃之间独立),更具体地,她会吃 $[l, r]$ 中所有的甜甜圈,小忍认为甜甜圈第 $k$ 维美味度最重要,其次是第 $k - 1$ 维,依次类推,她会按降序吃,但排序让她觉得麻烦,所以她让学 OI 的你帮她打了个 LSD 基数排序

\vspace{5mm}
LSD 基数排序:将待排序的元素拆分为 $k$ 个关键字,然后先对所有元素的第 $k$ 关键字进行稳定排序,再对所有元素的第 $k - 1$ 关键字进行稳定排序,再对所有元素 的第 $k - 2$ 关键字进行稳定排序...最后对所有元素的第 $1$ 关键字进行稳定排序,这样就完成了对整个待排序序列的稳定排序,完成后,各元素按照从第 $1$ 关键字最重要,第 $2$ 关键字其次...这样的顺序排好

warning:本题中关键字的重要度是第 $k$ 维最重要,所以以上流程关键字顺序应该反过来
\vspace{5mm}

但你把这个任务交给了机房最菜的 Dyd ,他当然打挂了,更具体地,他的基排中有几维是用的不稳定排序,如果第 $i$ 维排序不稳定,排它时小于 $i$ 的维数就会被随机打乱(但它这一维保证有序),而小忍这次吃甜甜圈的丢脸值就是每个甜甜圈的排名乘以其丢脸系数的代数和,即 $\sum rk[i] * v[i]$  , $rk[i] \in [1, r - l + 1]$ 表示第 $i$ 个甜甜圈的排名

现在,对于每次吃甜甜圈,垃圾君想知道丢脸值的期望

\vspace{5mm}
简化题意

给定一个长度为 $n$ 的序列,其中每个元素有一个 $k$ 维权值和一个系数 $v_i$ ,完成以下操作:

1. 修改一个点的权值

2. 给定区间 $[l, r]$ ,将区间内的数进行基数排序,且给定的维度不稳定,求 $\sum rk[i] * v[i]$ 的期望,其中 $rk[i] \in [1, r - l + 1]$ 表示第 $i$ 个甜甜圈的排名

\subsection{输入}
第一行两个整数 $n, k$ ,表示甜甜圈个数,甜甜圈美味的维数

接下来一行 $n$ 个整数描述每个甜甜圈的丢脸系数 $v[i]$ 

接下来 $n$ 行每行 $k$ 个整数 $a[1 \sim k]$ 描述一个甜甜圈

接下来一行一个整数 $m$ 表示事件

接下来 $m$ 行,每行第一个整数 $opt$ 表示事件类型( $opt \in [1, 2]$ ),接下来的输入视操作而定:

\begin{itemize}
    \item 如果 $opt = 1$ ,输入 $k + 1$ 个整数 $id$ 和 $b[1 \sim k]$ ,表示将第 $id$ 个甜甜圈换成美味度为 $b[1 \sim k]$ 的 
    \item 如果 $opt = 2$ ,输入三个整数 $l, r, num$ ,表示小忍吃的甜甜圈的编号区间为 $[l, r]$ ,且有 $num$ 维排序不稳定

    接下来一行 $num$ 个整数 $c[1 \sim num]$ ,表示不稳定的维度 
\end{itemize}

\subsection{输出}

对于每个事件 $2$ ,输出一行一个数表示丢脸值的期望,答案对 $998244353$ 取模

\subsection{输入输出样例1}
\subsubsection{输入样例}
\begin{verbatim}
8 5
1 2 3 4 5 6 7 8
1 2 3 4 5
2 3 4 1 7
1 2 3 8 1
1 1 1 1 9
9 9 9 9 1
3 3 2 1 4
5 4 3 2 1
7 5 2 2 1
5
1 6 1 2 1 2 1
2 3 6 2
1 3
1 8 5 4 3 1 2
1 1 1 2 3 5 4
2 1 8 2
2 4
\end{verbatim}
\subsubsection{输出样例}
\begin{verbatim}
47
499122360
\end{verbatim}
\subsubsection{样例解释}
第一次询问,区间为 $[3, 6]$ ,第 $1, 3$ 维的排序不稳定,但这不影响区间的排序情况,只有一种排序结果,其 $rk$ 数组为 $rk[3 \sim 6] = \{3, 1, 2, 4\}$ 

第二次询问,区间为 $[1, 8]$ ,第 $2, 4$ 维的排序不稳定,导致最后的排序情况有两种,其 $rk$ 数组为 $rk[1 \sim 8] = \{3, 2, 6, 1, 5, 7, 8, 4\}$ 或者 $rk[1 \sim 8] = \{3, 2, 6, 1, 5, 8, 7, 4\}$ 则期望为 $Ans = \frac{1}{2} * 184 + \frac{1}{2} * 183 \equiv 499122360 \pmod {998244353}$
\subsection{输入输出样例2, 3, 4}
见下发文件
\subsection{约定和数据范围}
\begin{center}
\begin{tabular}{*{3}{|C{9em}}|}
\hline
    测试点编号 & 特殊范围 & 特殊性质 \\ \hline
    1, 2 &$n, m \le 100$ & 无 \\ \hline
    3, 4 & $k = 1$ & 无  \\ \hline
    5, 6 & 无 & 不存在修改操作 \\ \hline
    7 $\sim$ 10 & 无 & 无 \\ \hline
\end{tabular}
\end{center}
对于所有数据, $n, m \le 10^4, k \le 100$ , $1 \le v, a, b \le 10^9$ , $\sum num \le 10^6$ , $c_i \le k$  , $1 \le l \le r \le n$
\begin{figure}[h]
    \centering
    \includegraphics[scale=0.25]{pic/Ah2xdcQ4nNvr73g.jpg}
    \caption{物语}
    \label{fig:enter-label}
\end{figure}
\end{document}